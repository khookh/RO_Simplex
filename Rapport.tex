\documentclass[12pt, openany]{report}
\usepackage[utf8]{inputenc}
\linespread{1.5}
\usepackage{biblatex}
\usepackage[T1]{fontenc}
\usepackage[a4paper,left=2cm,right=2cm,top=2cm,bottom=2cm]{geometry}
\usepackage[english]{babel}
\usepackage{libertine,multirow}
\usepackage{listings}
\usepackage{amsfonts}        % Police math
\usepackage{graphicx}        % Images
\usepackage{mathrsfs}        % Police math
\usepackage{amsmath}         % Math
\usepackage{amstext}         % Texte
\usepackage{tikz}            % Graphiques
\usepackage{color}           % Couleurs
\usepackage{tabularx}        % Tableaux avancés
\usepackage{siunitx}         % Unités
\lstset{
  aboveskip=3mm,
  belowskip=-2mm,
  backgroundcolor=\color{white},
  basicstyle=\footnotesize,
  breakatwhitespace=false,
  breaklines=true,
  captionpos=b,
  commentstyle=\color{red},
  deletekeywords={...},
  escapeinside={\%*}{*)},
  extendedchars=true,
  framexleftmargin=16pt,
  framextopmargin=3pt,
  framexbottommargin=6pt,
  frame=tb,
  keepspaces=true,
  keywordstyle=\color{blue},
  language=C,
  literate=
  {²}{{\textsuperscript{2}}}1
  {⁴}{{\textsuperscript{4}}}1
  {⁶}{{\textsuperscript{6}}}1
  {⁸}{{\textsuperscript{8}}}1
  {€}{{\euro{}}}1
  {é}{{\'e}}1
  {è}{{\`{e}}}1
  {ê}{{\^{e}}}1
  {ë}{{\¨{e}}}1
  {É}{{\'{E}}}1
  {Ê}{{\^{E}}}1
  {û}{{\^{u}}}1
  {ù}{{\`{u}}}1
  {â}{{\^{a}}}1
  {à}{{\`{a}}}1
  {á}{{\'{a}}}1
  {ã}{{\~{a}}}1
  {Á}{{\'{A}}}1
  {Â}{{\^{A}}}1
  {Ã}{{\~{A}}}1
  {ç}{{\c{c}}}1
  {Ç}{{\c{C}}}1
  {õ}{{\~{o}}}1
  {ó}{{\'{o}}}1
  {ô}{{\^{o}}}1
  {Õ}{{\~{O}}}1
  {Ó}{{\'{O}}}1
  {Ô}{{\^{O}}}1
  {î}{{\^{i}}}1
  {Î}{{\^{I}}}1
  {í}{{\'{i}}}1
  {Í}{{\~{Í}}}1,
  morekeywords={*,...},
  numbers=left,
  numbersep=10pt,
  numberstyle=\tiny\color{black},
  rulecolor=\color{black},
  showspaces=false,
  showstringspaces=false,
  showtabs=false,
  stepnumber=1,
  stringstyle=\color{gray},
  tabsize=4,
  title=\lstname,
}
\begin{document}
\linespread{1.25}
  \begin{sffamily}
  \begin{center}
    
	\textsc{\Large INFO-F310 - Algorithmique et recherche opérationnelle}\\[1.0cm]
    \vspace{100pt}
    % Title
  
    { \huge \bfseries Script Python : Programmation linéaire\\[0.4cm] }

	\centering
	\vspace{330pt}
    % Author and supervisor
    \begin{minipage}{0.4\textwidth}
      \begin{flushleft} \large
      \textit{}\\
        
        Stefano \textsc{Donne}\\
        matricule : 408801
        
      
      \end{flushleft}
    \end{minipage}
    \begin{minipage}{0.4\textwidth}
      \begin{flushright} \large
     
      \end{flushright}
    \end{minipage}

    \vfill

    % Bottom of the page
    {\large 11 Avril 2020}

  \end{center}
  \end{sffamily}

\newpage

\chapter{Programme linéaire}
\noindent
Il faut en premier lieu modéliser ce problème. Cela est fait à l'aide de la programmation linéaire; c'est un modèle mathématique dans lequel la fonction objectif et les contraintes sont linéaires en les variables.

\textbf{Indices :}\\
$I = \{1,...,n\}$ Panneaux \\
$J = \{1,...,m\}$ Planches

\textbf{Constantes :}\\
$N_{i}$ = nombre de panneaux de longueur $l_{i}$ à découper\\
$M$ = nombre de planches de longueur $L$ disponibles\\
$l_{i}$ = longueur des panneaux d'indice $i$\\
$L$ = longueur des planches disponibles

\textbf{Variables de décision :}\\
$x_{ij}$ = nombres de panneaux d'indice $i$ découpés dans la planche d'indice $j$\\
$C_{j}$ = choix d'utiliser pour découpe la planche d'indice $j$ ou non (booléen)\\
\bigskip
\noindent 
\textbf{Et la formulation suivante}\\
 \begin{minipage}{0.8\textwidth}
 \begin{center}
    \textbf{min} \; $\displaystyle\sum_{j \in J} C_{j}$ \\
    \textbf{s.t.} \; $x_{ij} \geq 0 \quad \hfill \forall i \in I, \forall j \in J $ \hfill (C1) \\
     $ C_{j} \in \{0,1\} \quad \hfill \forall j \in J$  \hfill(C2) \\
     $\displaystyle\sum_{j \in J} C_{j} \leq M \quad \hfill \forall j \in J$  \hfill(C3) \\
     $\displaystyle\sum_{j \in J} x_{ij} = N_{i} \quad \hfill \forall i \in I$  \hfill(C4) \\
     $\displaystyle\sum_{i \in I} x_{ij} . l_{i} \leq L . C_{j} \; \hfill \forall j \in J$   \hfill(C5) \\
 \end{center}
 \end{minipage}
\newpage

\noindent 
\\ \textbf{C1 :} Il ne peut pas y avoir un nombre négatif de panneaux coupés. 
\\ \textbf{C2 :} Une planche peut être sélectionnée \textit{(1)} et servir à la découpe de panneaux, ou être inutilisée \textit{(0)}.
\\ \textbf{C3 :} La somme des planches utilisées ne peut pas dépasser le nombre de planches disponibles.
\\ \textbf{C4 :} La somme sur toutes les planches des panneaux d'indice $i$ découpés doit être strictement égale au nombre de panneaux nécessaires.
\\ \textbf{C5 :} Soit la planche $j$ est choisie pour découpe \textit{($C_{j}=1$)} et dans ce cas la somme des longueurs des panneaux découpés ne peut dépasser la longueur $L$ de la planche. Soit la planche n'est pas choisie \textit{($C_{j}=0$)} et en adéquation avec la contrainte C1 il est impossible de découper des panneaux sur la planche (trivial).
\\
\\ \textbf{Fonction objectif :} Minimise le nombre de planches utilisées pour la découpe.

\newpage
\chapter{Génération du fichier CPLEX LP}

Les constantes du problème sont données sous forme d'un fichier \textit{instance} dont le format est décrit dans l'énoncé du projet.
Il est donné sous la forme suivante : \\ \\
\begin{minipage}{1\textwidth}
\begin{center}
$4.0$ \quad $8$ \\
$1.5$ \quad $4$ \\
$0.75$ \quad $6$ \\
$0.22$ \quad $12$ \\
\end{center}
\end{minipage}
\\ \\
Le script python \textit{generate\_lp\_instance} récupère le chemin vers le fichier d'instance qui lui a été donné en argument. Les constantes du problème sont extraites de ce fichier et stockées dans une liste. 
La fonction \textit{generate\_cplex} sert à proprement parler à générer le fichier \textit{CPLEX LP} et à le remplir en suivant les conventions liées au language. Le fichier qui a été généré à partir de l'instance est visible ci-dessous .

\begin{lstlisting}
Minimize
    obj: C_1 + C_2 + C_3 + C_4 + C_5 + C_6 + C_7 + C_8
Subject To
    cst_l_1: 1.5 X_1_1 + 0.75 X_2_1 + 0.22 X_3_1 - 4.0 C_1 <= 0
    cst_l_2: 1.5 X_1_2 + 0.75 X_2_2 + 0.22 X_3_2 - 4.0 C_2 <= 0
    cst_l_3: 1.5 X_1_3 + 0.75 X_2_3 + 0.22 X_3_3 - 4.0 C_3 <= 0
    cst_l_4: 1.5 X_1_4 + 0.75 X_2_4 + 0.22 X_3_4 - 4.0 C_4 <= 0
    cst_l_5: 1.5 X_1_5 + 0.75 X_2_5 + 0.22 X_3_5 - 4.0 C_5 <= 0
    cst_l_6: 1.5 X_1_6 + 0.75 X_2_6 + 0.22 X_3_6 - 4.0 C_6 <= 0
    cst_l_7: 1.5 X_1_7 + 0.75 X_2_7 + 0.22 X_3_7 - 4.0 C_7 <= 0
    cst_l_8: 1.5 X_1_8 + 0.75 X_2_8 + 0.22 X_3_8 - 4.0 C_8 <= 0
    cst_s_1:  X_1_1 + X_1_2 + X_1_3 + X_1_4 + X_1_5 + X_1_6 + X_1_7 + X_1_8 = 4
    cst_s_2:  X_2_1 + X_2_2 + X_2_3 + X_2_4 + X_2_5 + X_2_6 + X_2_7 + X_2_8 = 6
    cst_s_3:  X_3_1 + X_3_2 + X_3_3 + X_3_4 + X_3_5 + X_3_6 + X_3_7 + X_3_8 = 12
    cst_cj: C_1 + C_2 + C_3 + C_4 + C_5 + C_6 + C_7 + C_8 <= 8
Bounds
    X_1_1 >= 0
    X_1_2 >= 0
    X_1_3 >= 0
    X_1_4 >= 0
    X_1_5 >= 0
    X_1_6 >= 0
    X_1_7 >= 0
    X_1_8 >= 0
    X_2_1 >= 0
    X_2_2 >= 0
    X_2_3 >= 0
    X_2_4 >= 0
    X_2_5 >= 0
    X_2_6 >= 0
    X_2_7 >= 0
    X_2_8 >= 0
    X_3_1 >= 0
    X_3_2 >= 0
    X_3_3 >= 0
    X_3_4 >= 0
    X_3_5 >= 0
    X_3_6 >= 0
    X_3_7 >= 0
    X_3_8 >= 0
Integer
    X_1_1
    X_1_2
    X_1_3
    X_1_4
    X_1_5
    X_1_6
    X_1_7
    X_1_8
    X_2_1
    X_2_2
    X_2_3
    X_2_4
    X_2_5
    X_2_6
    X_2_7
    X_2_8
    X_3_1
    X_3_2
    X_3_3
    X_3_4
    X_3_5
    X_3_6
    X_3_7
    X_3_8
Binary
    C_1
    C_2
    C_3
    C_4
    C_5
    C_6
    C_7
    C_8

\end{lstlisting}
Le nom des variables respecte la convention utilisée lors de la modélisation sous forme de programme linéaire. \\ \\
\begin{minipage}{0.8\textwidth}
\begin{flushleft}
 X\_i\_j = $x_{ij}$ \\
 C\_j = $C_{j}$ \\
 cst\_l\_j = contrainte C5\\
 cst\_s\_i = contrainte C4\\
 cst\_cj = contrainte C3\\
 \end{flushleft}
\end{minipage}

\newpage
\chapter{Log de GLPK}
Une fois le fichier \textit{CPLEX} généré, il est traité par le solveur GLPK à l'aide de la commande \textit{(sous linux)} $\$\;glpsol --lp\;CPLEX.lp\;-o\;log.sol$ .
Cela génère un fichier log de GLPK appelé ici \textit{log.sol}, il est retranscrit ci-dessous.

\begin{lstlisting}
Problem:    
Rows:       12
Columns:    32 (32 integer, 8 binary)
Non-zeros:  64
Status:     INTEGER OPTIMAL
Objective:  obj = 4 (MINimum)

   No.   Row name        Activity     Lower bound   Upper bound
------ ------------    ------------- ------------- -------------
     1 cst_l_1                    -1                           0 
     2 cst_l_2                    -1                           0 
     3 cst_l_3                 -0.25                           0 
     4 cst_l_4                     0                           0 
     5 cst_l_5                     0                           0 
     6 cst_l_6                     0                           0 
     7 cst_l_7                     0                           0 
     8 cst_l_8                 -0.61                           0 
     9 cst_s_1                     4             4             = 
    10 cst_s_2                     6             6             = 
    11 cst_s_3                    12            12             = 
    12 cst_cj                      4                           8 

   No. Column name       Activity     Lower bound   Upper bound
------ ------------    ------------- ------------- -------------
     1 C_1          *              1             0             1 
     2 C_2          *              1             0             1 
     3 C_3          *              1             0             1 
     4 C_4          *              0             0             1 
     5 C_5          *              0             0             1 
     6 C_6          *              0             0             1 
     7 C_7          *              0             0             1 
     8 C_8          *              1             0             1 
     9 X_1_1        *              2             0               
    10 X_2_1        *              0             0               
    11 X_3_1        *              0             0               
    12 X_1_2        *              2             0               
    13 X_2_2        *              0             0               
    14 X_3_2        *              0             0               
    15 X_1_3        *              0             0               
    16 X_2_3        *              5             0               
    17 X_3_3        *              0             0               
    18 X_1_4        *              0             0               
    19 X_2_4        *              0             0               
    20 X_3_4        *              0             0               
    21 X_1_5        *              0             0               
    22 X_2_5        *              0             0               
    23 X_3_5        *              0             0               
    24 X_1_6        *              0             0               
    25 X_2_6        *              0             0               
    26 X_3_6        *              0             0               
    27 X_1_7        *              0             0               
    28 X_2_7        *              0             0               
    29 X_3_7        *              0             0               
    30 X_1_8        *              0             0               
    31 X_2_8        *              1             0               
    32 X_3_8        *             12             0               

Integer feasibility conditions:

KKT.PE: max.abs.err = 0.00e+00 on row 0
        max.rel.err = 0.00e+00 on row 0
        High quality

KKT.PB: max.abs.err = 0.00e+00 on row 0
        max.rel.err = 0.00e+00 on row 0
        High quality

End of output

\end{lstlisting}
Le solveur a déterminé que le nombre \textbf{minimum} de planches nécessaires pour découper les panneaux requis est de \textbf{4}. La colonne Activity correspondante aux X\_i\_j donne le détail de la répartition des panneaux. \\
Avec dans la première planche deux panneaux de 1.5m de long découpés. \\
Dans la seconde à nouveau deux panneaux de 1.5m de long découpés. \\
Dans la troisième 5 panneaux de 0.75m de long. \\
Et enfin dans la dernière planche 12 panneaux de 0.22m de long et 1 panneau de 0.75m de long. \\
La colonne activity liée aux contraintes nous montre qu'elles ont toutes été respectées. Les cst\_l\_j sont toutes <= 0, ce qui signifie que dans aucune planche utilisée il n'est découpé une longueur de panneaux > $L$. 
Les cst\_s\_i sont elles toutes égales à $N_{i}$, par conséquent le nombre de panneaux d'indice $i$ découpés est strictement égal au nombre de panneaux nécessaires.
\end{document}
